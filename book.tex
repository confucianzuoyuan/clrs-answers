\documentclass[cn,11pt,chinese]{elegantbook}

\usepackage{diagbox}
\usepackage{minted}
\usepackage{minted,tabularx,tikz}
\usepackage{graphicx}
\usetikzlibrary{
  shapes,
  shapes.geometric,
  decorations.text,
  shapes.geometric,
  calc,
  decorations.pathreplacing,
  automata,
  positioning,
  arrows
}

\usepackage{tcolorbox}
\tcbuselibrary{skins, breakable, theorems}

\setcounter{tocdepth}{2}

\title{尚硅谷Flink教程}

% 本文档命令
\usepackage{array}
\newcommand{\ccr}[1]{\makecell{{\color{#1}\rule{1cm}{1cm}}}}

\begin{document}

\pagestyle{empty}

\begin{tikzpicture}[remember picture,overlay]
%%%%%%%%%%%%%%%%%%%% Background %%%%%%%%%%%%%%%%%%%%%%%%
\fill[Dandelion] (current page.south west) rectangle (current page.north east);




%%%%%%%%%%%%%%%%%%%% Background Polygon %%%%%%%%%%%%%%%%%%%%

\foreach \i in {2.5,...,22}
{
    \node[rounded corners,Dandelion!60,draw,regular polygon,regular polygon sides=6, minimum size=\i cm,ultra thick] at ($(current page.west)+(2.5,-5)$) {} ;
}

\foreach \i in {0.5,...,22}
{
\node[rounded corners,Dandelion!60,draw,regular polygon,regular polygon sides=6, minimum size=\i cm,ultra thick] at ($(current page.north west)+(2.5,0)$) {} ;
}

\foreach \i in {0.5,...,22}
{
\node[rounded corners,Dandelion!90,draw,regular polygon,regular polygon sides=6, minimum size=\i cm,ultra thick] at ($(current page.north east)+(0,-9.5)$) {} ;
}


\foreach \i in {21,...,6}
{
\node[Dandelion!85,rounded corners,draw,regular polygon,regular polygon sides=6, minimum size=\i cm,ultra thick] at ($(current page.south east)+(-0.2,-0.45)$) {} ;
}


%%%%%%%%%%%%%%%%%%%% Title of the Report %%%%%%%%%%%%%%%%%%%% 
\node[left,black,minimum width=0.625*\paperwidth,minimum height=3cm, rounded corners] at ($(current page.north east)+(0,-9.5)$)
{
{\fontsize{25}{30} \selectfont \bfseries 算法导论习题解答}
};

%%%%%%%%%%%%%%%%%%%% Author Name %%%%%%%%%%%%%%%%%%%% 
\node[left,black,minimum width=0.625*\paperwidth,minimum height=2cm, rounded corners] at ($(current page.north east)+(0,-13)$)
{
{\Large \textsc{左元}}
};

%%%%%%%%%%%%%%%%%%%% Year %%%%%%%%%%%%%%%%%%%% 
\node[rounded corners,fill=Dandelion!70,text =black,regular polygon,regular polygon sides=6, minimum size=2.5 cm,inner sep=0,ultra thick] at ($(current page.west)+(2.5,-5)$) {\LARGE \bfseries 2022};

\end{tikzpicture}

\frontmatter

\tableofcontents
%\listofchanges

\mainmatter

\chapter{算法在计算中的作用}

\section{算法}

\section{作为一种技术的算法}

\section{思考题}

\chapter{算法基础}

\section{插入排序}

\section{分析算法}

\section{设计算法}

\section{思考题}

\chapter{数论算法}

\section{基础数论概念}

\subsection{练习31.1-1}

\begin{tcolorbox}[colback = red!25!white, colframe = red!75!black]
  证明:若$a > b > 0$,且$c = a + b$,则$c \mod a = b$。
\end{tcolorbox}

\begin{tcolorbox}[colback = green!25!white, colframe = green!75!black]
  $$
  \begin{aligned}
  c \mod a & = (a + b) \mod a \\
           & = (a \mod a) + (b \mod a) \\
           & = 0 + b \\
           & = b.
  \end{aligned}
  $$
\end{tcolorbox}

\subsection{练习31.1-2}

\begin{tcolorbox}[colback = red!25!white, colframe = red!75!black]
  证明有无穷多个素数。(提示:证明素数$p_1,p_2,\cdots,p_k$都不能整除$(p_1 p_2 \cdots p_k) + 1$。)
\end{tcolorbox}

\begin{tcolorbox}[colback = green!25!white, colframe = green!75!black]
  $$
  \begin{aligned}
      ((p_1 p_2 \cdots p_k) + 1) \mod p_i
  & = (p_1 p_2 \cdots p_k) \mod p_i + (1 \mod p_i) \\
  & = 0 + 1 \\
  & = 1.
  \end{aligned}
  $$

  如果 $p_1 p_2 \cdots p_k$ 已经是所有的素数的话,那么 $(p_1 p_2 \cdots p_k) + 1$ 又是一个新的素数。
\end{tcolorbox}

\subsection{练习31.1-3}

\begin{tcolorbox}[colback = red!25!white, colframe = red!75!black]
  证明:如果 $a \mid b$ 且 $b \mid c$,则$a \mid c$。
\end{tcolorbox}

\begin{tcolorbox}[colback = green!25!white, colframe = green!75!black]
  \begin{itemize}
    \item 如果$a \mid b$,那么$b = a \cdot k_1$。
    \item 如果$b \mid c$,那么$c = b \cdot k_2 = a \cdot (k_1 \cdot k_2) = a \cdot k_3$,那么$a \mid c$。
  \end{itemize}
\end{tcolorbox}


\end{document}